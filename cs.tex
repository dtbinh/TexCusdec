\documentclass[a4paper,10pt, english]{article}

\usepackage{amssymb}
\usepackage{amsmath}
\usepackage{enumitem}
\usepackage{graphicx}
\usepackage{subfig}
\usepackage{amsthm}




\newcommand{\D}{\displaystyle}

\newtheorem{theo}{Theorem}[section]
\newtheorem{dfn}{Definition}[section]
\newcommand{\R}{\mathbb{R}}
\newtheorem{prop}[theo]{Proposition}



















\begin{document}


\title{Decentrilized control for Cucker Smale system}
\author{}
\maketitle












\section{Introduction}
The goal of this work is to design a decentralized control strategy for the Cucker-Smale system to steer the system to consensus while preserving the spatial connectivity of the initial interaction graph.

Consider the Cucker-Smale system
\begin{align}
\begin{cases}
\D
\dot{x_i} = v_i,\\
\dot{v_i} = \frac{1}{N}\sum_{i=1}^{N}a_{ij}(v_j - v_i) + u_i, \\
x(0) = x_0,
v(0) = v_0.
\end{cases}
\label{csm}
\end{align}
where $i=1,\dots N$, $N$ is a number of $d$-dimensional agents, a pair $(x_i, v_i)\in \mathbb{R}^{2d}$, represents the position and velocity of every agent, $u_i$ is an external controller,  $a_{ij} = a(\|x_i - x_j\|)$,  $a:[0, \infty)\longrightarrow [0, \infty)$ is an interaction function to be suitably defined later.

We would like to design a decentralized optimal control strategy to steer the system (\ref{csm}) to consensus by minimizing the agents' velocity deviation throughout the system evolution.
\begin{equation}
J(x, v) = \int_{0}^{T}B(v, v):= \frac{1}{2N^2}\int_{0}^{T}\sum_{i, j = 1}^{N}\|v_i - v_j\|^2.
\label{Vt}
\end{equation}


To introduce decentralization let us define the dynamic graph for the interaction between agents.
\begin{dfn}
We call $\mathcal{G}(t) =  (\mathcal{V}, \mathcal{E}(t) )$ a dynamic graph consisting of a set of vertices $\mathcal{V} = \{1, \dots,  N\}$ indexed by the set of agents and a time varying set of links 
$\mathcal{E}(t) = \{(i, j)|i, j \in \mathcal{V}\}$ such that, for any $R>0$
$$
\mbox{if}\quad 0<\|x_i(t) - x_j(t)\| \leq R \quad\mbox{then},\quad (i, j)\in \mathcal{E}(t).
$$
\end{dfn}
And let us denote $\mathcal{N}_i(t) = \{j|(i, j)\in\mathcal{E}(t)\}$ the neighbourhood of agent i.
Let us also denote an agent along with its neighbours as a set $\bar{\mathcal{N}}_i(t)$.



The decentralization strategy consists in decomposition the state domain in sub-domains for each agent $i$ and his neighbours $\mathcal{N}_i(t)$ where  an optimal control is solved independently to obtain the suitable decentralized control for this agent. Such decentralization have already been utilized for first order systems in [jadbabai].

 The algorithm adopts an model predictive control approach by partitioning the time interval $[0, T]$ into $n$ time windows of length $\Delta T$: $0 = T_0, T_1, T_2, \dots, T_m = T$ and for each time window  $[T_{l-1}, T_{l}]$, $l = 1, \dots, m$ for each agent $p$, $p = 1, \dots, N$  solving the following optimal control problem
\begin{equation}
 \min_{x, v} J_p(x, v) = \int_{0}^{T}B(v, v) + \int_{0}^{T}\sum_{j\in \mathcal{N}_p(t)} V(\|x_p - x_j\|),
\label{opci}
\end{equation}
subject to (\ref{csm}) where $i \in \bar{\mathcal{N}}_p(T_l)$,  $N = N_p:= \sharp \bar{\mathcal{N}}_p(T_l)$, with initial conditions taken as endpoints form the solution on the previous time window $l-1$,  and $V(r)$ is an artificial potential function  as was used in [jadbabai] to maintain the connectivity of the initial interaction graph $\mathcal{G}(0)$ 
\begin{equation}
 V(r) = \frac{1}{r^2} + \frac{1}{R^2 - r^2}, \qquad 0 < r < R,
\label{potential}
\end{equation}
this potential not only allows for maintaining links in the network but also for avoiding collisions among the agents. Note, that we are only interested in maintaining the initial connectivity of the graph which rules of the risk for the potential to grow unbound.

 Therefore, at each time window $l$ we obtain for each agent $p$ a suitable decentralized control $u_p(\mathcal{N}_p(T_l))$ which only depends on its neighbours in the time window $l$.
 
 
 \section{The optimal contro problem and its discretization}
 
 We represent the optimal control problem (\ref{opci}) for the agent $p$ on a time window $T_l$ in Mayer form.
 
 
 \begin{align}
 \D
 \mbox{minimize}\quad  C(X(T)),\\
  \mbox{subject to} \quad \dot{X} = F(X(t), u(t)),
  X(0) = X_0,
 \label{opcm}
 \end{align}
 where $X\in W^{1, \infty}$, $u\in L^{\infty}$, the control $u \in R^{N_p\times d}$,
 the state $X\in \mathbb{R}^{N_p \times d\times N_p\times d \times 1}$ is a compact form composed of the position and velocity 
states of (\ref{csm})   $X(t) = (x(t), v(t), z(t))$, $x(t) = (x_{i_1}(t), \dots, x_{i_{N_p}}(t))$,
 $v(t) = (v_{i_1}(t), \dots, v_{i_{N_p}}(t))$, $\{i_1, \dots, i_{N_p}\} \in \bar{\mathcal{N}}_p(T_l)$, 
 and $z(t)$ is an auxiliary component 
 $$
 z(t) = B(v, v) + \sum_{j\in \mathcal{N}_p(t)} V(\|x_p - x_j\|),
 $$
 and $F(X)$ is the dynamics of the system (\ref{csm}) in compact form
 $$
 F(X(t), u) =
  \left( 
  \begin{array}{c}
  v\\
   - L_xv + u\\
   z\\
 \end{array} 
 \right), 
 $$
 where $L_x$ is the Laplacian of the $N_p\times N_p$ matrix $(a(\|x_i - x_j\|)/N_p)_{i, j\in\bar{\mathcal{N}}_p(T_l)}$ and depends on $x$.
 
 
 Note, since our problem is autonomous and all time windows are of the same width without loss of generality
 we can transform the problem on time interval $[0, \Delta T]$. 
 To discretize the problem we utilize a Runge-Kutta scheme developed in [hager] on a uniform time mesh and the following time-step size
 \begin{equation}
   h = \frac{\Delta T}{n},
   \label{h}
 \end{equation}
 where $n$ is the total number of discrete time intervals in $[0, \Delta T]$.  The value of $X(t)$ at the discrete time $t_k$ is denoted with
 $$
 X_k = X(t_k), \qquad t_k = k \quad\mbox{for} \quad k = 1, \dots n.
 $$
 We consider an $s$-stage Runge-Kutta discretization scheme defined by setting the coefficients $a_{ij}$ and
 $b_{i}$, $1\leq i, j\leq s$, such that they satisfy conditions given in [hager].
 
 Corresponding to the discretezation setting, the optimal control problem (\ref{opcm})  attains the following form
 
 \begin{align}
  \D
  \mbox{minimize}\quad C(X_n),\\
   \mbox{subject to} \quad X_{k+1} = X_{k} + h\sum_{i=1}^{s}b_iF(y_i, u_{ki}), \qquad X_0 = X(0),\\
	y_i = X_k + h\sum_{j=1}^{s}a_{ij}F(y_j, u_{kj}),
  \label{opcmd}
  \end{align}
 
 for $1\leq i, j\leq s$, and $0\leq k\leq n-1$.
 Where the vectors $y_j$ and $u_{kj}$ are intermediate state and control variables on the interval $[t_k, t_{k+1}]$. In such representatation the state uniqueness is guaranteed see [hager]. 
Following [hager] the optimality system corresponding to (\ref{opcmd}) is given by

  \begin{align}
   \D
   \mbox{minimize}\quad C(X_n),\\
    \mbox{subject to} \quad X_{k+1} = X_{k} + h\sum_{i=1}^{s}b_iF(y_{ki}, u_{ki}), \qquad X_0 = X(0),\\
 	y_i = X_k + h\sum_{j=1}^{s}a_{ij}F(y_{kj}, u_{kj}),\\
 	p_k = p_{k+1} + \sum_{i=1}^{s}\psi_{ki}, \qquad p_n = - \nabla_xC(X_n),\\
 	\psi_{ki} = (\Delta_x F(y_{ki}, u_{ki}))^{T} \left( p_{k+1} + \sum_{j=1}^{s} \frac{a_{ji}}{b_i}\psi_{kj}\right), 
   \label{opsystem}
   \end{align}

From this system the following gradient results
$$
\nabla_{u_{ki}} C(u_{ki}) = - (\Delta_u F(y_{ki}, u_{ki}))^{T} \left( p_{k+1} + \sum_{j=1}^{s}\frac{a_{ji}}{b_i}\psi_{kj}\right), 
$$

 for $1\leq i, j\leq s$, and $0\leq k\leq n-1$.
 The well-posedness of the optimal control problem (\ref{opsystem}) and error estimates in the discrete approximation are present in [hager].
 
 For the minimization procedure the nonlinear conjugate gradient  method developed in [hager zhang] is deployed.
 This scheme satisfies the decent condition $\|\nabla_{u_{ki}} C(u_{ki})\|d_k \leq -\frac{7}{8}\|\nabla_{u_{ki}} C(u_{ki})\|^2$, where $d_k$ is an NCG decent direction. Our choice is motivated by our numerical experiments. In fact, the Hager-Zhang NCG scheme results to be the most efficient among the known formulas.
 
 \section{The dicentralized MPC strategy}
 
 The algorithm reads the following: given the system (\ref{csm}):
  
 \begin{enumerate}
   \item Partition time interval $[0, T]$ into time windows $\Delta T$: $0 = T_0, T_1, \dots, T_m = T$.
   \item For each time window  $[T_{l-1}, T_{l}]$, $l = 1, \dots, m$ and for each agent $p$, $p = 1, \dots, N$  solve the problem (\ref{opcm}) with the initial conditions $(x^l_p(nh), v^l_p(nh))$, for $l = 0$ take $(x_{0p}, v_{0p})$.
   \item From the solution of (\ref{opcm}) retrieve the discrete state for the agent $p$ in the window $l$ $(x^l_p(kh), v^l_p(kh))$ and control $u^l_p(kh)$, $k = 1, \dots, n$, where $h$ as in (\ref{h}) 
   \item Repeat the steps 2-3 until the solutions for all windows are obtained.
 \end{enumerate}
 
 


\end{document}
